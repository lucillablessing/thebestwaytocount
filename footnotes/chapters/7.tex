\documentclass[../footnotes.tex]{subfiles}

\begin{document}

\mychapter{chapter seven}

\myfootnote{} tally marks are based on the principle that you count only by drawing additional strokes. this doesn't work with normal binary counting, which requires you to erase strokes. instead, binary tally marks work using a zigzag-like fractal pattern. imagine you have a vertical binary number, with the least significant bit on the bottom. to count 1, draw a tiny line segment that ends on that imaginary bottom bit. to count from $n$ to $n + 1$, nudge your imaginary binary number to the right, then draw a line from where you currently are to the most significant bit that has flipped. here are some examples, where a digit represents the height of a local peak: 0, 01, 010, 0102, 01020, 010201, 0102010, 01020103 (\myurl{https://oeis.org/A007814}). thus every next line will alternate between ascending and descending, and counting up to higher powers of two makes local peaks that are higher and higher. every odd number ends up on the bottom, while every even number ends up on an unresolved peak. the pattern for counting to one power of two looks like a scaled-up version of the pattern for the previous power of two, making this pattern a fractal. all lines are connected, making this system of tally marks maximally efficient, as you never have to lift your pen. to read a binary number from this system, first find the largest power of two and write a 1 for it, then for every smaller power of two, if its rightmost instance is to the right of the previous power of two's rightmost instance, its bit is a 1; if to the left, it's a 0.

\myfootnote{} balanced ternary, with digits whose absolute value doesn't exceed 1, works in many ways like binary, and shares many advantages, such as simple arithmetic, good radix economy, and possibly even better finger counting if you're that dexterous. sadly not \emph{all} the advantages carry over, such as ultra-dense powers of the base letting you work at lots of different scales or have many factors be adjacent to a power of the base. and honestly, it being an odd base is kind of a death sentence, as it makes 2 need a digit sum test. still, it's probably the second best base after binary -- and \emph{definitely} the most usable odd base.

\myfootnote{} two's complement notation is a great way to write negative numbers with lots of really neat properties. it arises naturally from the ``$1 + 2 + 4 + 8 + \ldots = -1$'' pseudo-equality (let $1 + 2 + 4 + 8 + \ldots = S$, then $S$ has the property that $2 \times S = S - 1$, so $S = -1$), and extending it to the negative powers of two gives $\ldots 111.111 \ldots = 0$, thus $\mathord{\sim}x = -x$ (the formula ``$\mathord{\sim}x = -x - 1$'' in computing comes from ignoring the fractional part .000... turning into .111...). two's complement preserves modular arithmetic: $-7$ is ...111001, indicating that it's congruent to 1 modulo 8 (last three bits are 001). technically $n$'s complement works in all bases, but the leftmost recurring digit is always either ...000 or ...ZZZ, so binary is the only base that uses it ``to its full potential''.

\myfootnote{} we can compare a hypothetical binary system of money with a decimal and seximal one by looking at minimal sets of nominals that can be used to represent any possible value. the decimal one is inefficient because the nominals 1, 2, 5, and 10 are insufficient to represent all values from 1 to 20; you either need to double the 2 or double the 1. the seximal one is also inefficient for the opposite reason: to represent all possible values from 1 to 12, you can use the values 1, 2, 3, and 6, but then the 3 is redundant, already being the sum of 1 and 2. the results of these inefficiencies are also transparently visible in the value of the largest nominal in a ``complete'' set of eight: for binary, it's 128; for decimal, 50; for seximal, 72. (in addition, there's a mental math advantage; every nominal is just a power of two, so to make a transaction, you just spell out a number's bits. in contrast, something like $5 + 2 + 1 = 8$ is in no way made easier by using decimal.)

\myfootnote{} dividing the day into groups of 16 four times gives a binary second only slightly longer than the traditional sexagesimal second ($1/65536$ instead of $1/86400$ of a day). the in-between time units are also surprisingly useful: the ``binary hour'' and half of it are typical durations of lectures and lessons respectively; the ``binary maxime'', around 5 traditional minutes, is the ``square root of a day'' (there are as many maximes in a day as seconds in a maxime, namely 256) and is a typical small amount of time to ask someone to wait; the ``binary minute'', or 16 binary seconds, is the typical amount of time you can expect someone's attention span to last.

\myfootnote{} in music, a notehead can be followed by a dot to increase the note's duration by half, in other words, multiply it by $3/2$. that can be followed by another dot to increase it by another quarter, so overall multiplying by $7/4$, and another dot to increase it by another eighth, multiplying by $15/8$, and so on. these dots can be interpreted as spelling out the binary fractions 1.1, 1.11, and 1.111, for the numbers $3/2$, $7/4$, and $15/8$. by introducing a hollow dot for 0, you can create all sorts of binary fractions, like $5/4$, $13/8$, and $9/8$, which can otherwise only be represented by adding note values using tie bars -- and without making it any harder to sight-read, either, as you can just imagine the hollow dots as being placeholders for filled dots that have been omitted.

\myfootnote{} you can also prove that the exponents of Mersenne primes must themselves be prime more directly with number theory, without using binary representations: the differences of two powers $(a^n - b^n)$ can always be factorized as $(a - b)$ times something else; if the exponent is composite, say, $2^6 - 1$, we can represent it as $(2^3)^2 - 1^2$ and thus extract a factor of $2^3 - 1$. the only way to prevent such a factorization is if the exponent is prime -- and the base is two.

\myfootnote{} this too can be proved directly with a number-theoretic method. the \emph{sums} of two powers $(a^n + b^n)$, unlike differences, can only be factorized as $(a + b)$ times something else if the power is \emph{odd} (this has to do with how $-1$ is equal to all of its own odd powers, but not its even powers). so if the exponent has a nontrivial odd part, such as $2^6 + 1$, then it's possible to represent the number as a sum of two odd powers and factorize it: $(2^2)^3 + 1^3$ has a factor of $2^2 + 1$. this is impossible only if the exponent has no odd part -- in other words, it must be a power of two.

\myfootnote{} for just a few examples: they're the ``additive primes''; they're the sums of all entries in the rows of Pascal's triangle; they're precisely the numbers that cannot be written as sums of consecutive positive integers (``polite numbers''); they're the lexicographically earliest sequence of which no subset forms an arithmetic progression; they're a sequence whose first differences are the sequence itself; they're the number of elements in power sets, and are so crucial to the concept of power sets that one notation for the power set of $S$ is literally ``$2^S$''.

\myfootnote{} by the way, if you're an enthusiast of memorizing the circle constant's digits, then in binary you don't have to pick a side on the pi vs tau debate, since it amounts to just a bit shift (\myurl{https://tauday.com/tau-manifesto}). in fact, Bob Palais wrote in {\it Pi Is Wrong} (\myurl{http://math.utah.edu/\%7Epalais/pi.pdf}):

\begin{quoting}
	What really worries me is that the first thing we broadcast to the cosmos to demonstrate our ``intelligence,'' is $3.14 \ldots$ I am a bit concerned about what the lifeforms who receive it will do after they stop laughing at creatures who must rarely question orthodoxy. Since it is transmitted in binary, we can hope that they overlook what becomes merely a bit shift!
\end{quoting}

\myfootnote{} for those interested, Mathologer's video on ``Newton's calculus'' is excellent: \\
\myurl{https://youtu.be/4AuV93LOPcE}

\myfootnote{} this game is, of course, none other than ``2048''.

\myfootnote{} seximal powers of two (of course, you could \emph{instead} switch to a base where you \emph{don't} need to memorize the most important integer sequence...) \\
\myurl{https://seximal.net/math}

\myfootnote{} the dual number -- no, wait, \emph{this} one: \\
\myurl{https://en.wikipedia.org/wiki/Dual_number} \\
\myurl{https://en.wikipedia.org/wiki/Dual_(grammatical_number)}

\myfootnote{} ``but wait, aren't there twelve notes?'' glad you asked! yes, Western music does indeed use twelve notes per octave, but this has nothing to do with 12 being a highly composite number. rather, if we must reduce it to one reason, it's because $\log_2(3)$ is much better approximated by a fraction with denominator 12 (namely $19/12$) than any other small denominator; in number theory lingo, 12 is the denominator of a \emph{convergent} to $\log_2(3)$. if anything, it's highly \emph{unusual} for a highly composite number to show up here -- most other numbers of notes per octave famous for ``sounding good'' are primes or semiprimes (since the numerator and denominator of these convergents must be coprime). if you're curious, \emph{please} do read up more on this question because this answer doesn't even begin to do it justice. (in contrast to the prevalence of twelve notes in Western music specifically and few other cultures, binary octave equivalence seems to be far more widespread across musical traditions -- maybe because the ranges of people and instruments fail to overlap, so there's the need for some interval of equivalence; and $2:1$ is the most natural choice, being the simplest integer ratio after $1:1$.)

\myfootnote{} in decimal -- or seximal, for that matter -- there's a very clear dividing line between math where the powers of the base appear, which is always ``recreational'', and ``serious'' math, which always lacks them. we recognize that our base is really nothing more than an arbitrary convention, and ``objective'' math is just entirely devoid of these conventions. in contrast, powers of two permeate math in all its aspects, and let us reconcile what math itself finds important with what we find important. so if anything else other than numerological mysticism surrounding the number two, \emph{this} is what it means for base two to be natural.

\end{document}