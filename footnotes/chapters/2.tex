\documentclass[../footnotes.tex]{subfiles}

\begin{document}

\mychapter{chapter two}

\myfootnote{} an excellent introduction to finger binary by 3blue1brown: \\
\myurl{https://youtu.be/1SMmc9gQmHQ}

\myfootnote{} for example, in this base 6 proposal: \\
\myurl{http://shacktoms.org/base-six/base-six.htm}

\myfootnote{} strictly speaking, chisanbop is about pressing fingers down on a table, not extending or retracting them to or from a fist. but it can easily be reimagined this way.

\myfootnote{} or, actually, by this classification, the ``traditional decimal'' counting method is... actually undecimal? it lets you count eleven different numbers, 0 to 10 inclusive, the eleven possible values of an undecimal digit. guess we can be glad that inclusive counting shielded our ancestors from adapting base \emph{eleven}.

\myfootnote{} aside from just being a race about who can count highest, these differences have more practical applications. for example, in seximal you can do any one-digit addition with just finger counting. meanwhile, in binary, if you're so inclined, you can do \emph{five-bit} addition by performing bitwise operations on your fingers. it's extremely nerdy and really cool once you get the hang of it.

\myfootnote{} the proposal by Shack actually comes up with a seximal workaround: having the thumb touch the remaining four fingers in turn to represent 32, 33, 34, and 35, thus making each hand a niftimal digit. but this destroys the ability to quantize the counting into smaller digits than entire hands (base six \emph{squared}!), which themselves are now a weird amalgamation of binary and some leftovers -- and to what end? it only pushes the upper limit by an additional 25\%, compared to nearly tripling from seximal to chisanbop and increasing by over ten times from chisanbop to binary.

\myfootnote{} moreover, the regularity and repetition of binary counting makes the motions for finger binary extremely easy to pick up on. thanks to this, it's easily possible to count in binary automatically, not thinking about the intermediate steps and instead only interpreting the final result -- for instance to count words in a sentence -- thus reaping its rewards well before you've familiarized yourself with the ins and outs of binary mental math.

\myfootnote{} check out for instance any of these: \\
\myurl{https://youtu.be/zELAfmp3fXY} (wooden panels) \\
\myurl{https://youtu.be/4yBGbozevqs} (marble machine) \\
\myurl{https://youtu.be/--LolAtwecE} (toy train track) \\
\myurl{https://youtu.be/pXN1TH00HT8} (liquids) \\
\myurl{https://youtu.be/B3RC8UAd1Mo} (Cell Machine) \\
\myurl{https://youtu.be/KEhsMZlJFyU} (Baba Is You) \\
\myurl{https://youtu.be/iIaFcJYtluw} (Minecraft; only sticky pistons and observers)

\myfootnote{} Nystrom's quote: \\
\myurl{https://books.google.com/books?id=aNYGAAAAYAAJ}

\end{document}