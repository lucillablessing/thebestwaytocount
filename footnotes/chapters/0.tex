\documentclass[../footnotes.tex]{subfiles}

\begin{document}

\mychapter{chapter zero}

\myfootnote{} languages and writing systems convey information at about the same rate: \\
\myurl{https://pubmed.ncbi.nlm.nih.gov/22486107/} \\
\myurl{https://scientificamerican.com/article/fast-talkers/} \\
\myurl{https://content.time.com/time/health/article/0,8599,2091477,00.html} \\
\myurl{https://persquaremile.com/2011/12/21/which-reads-faster-chinese-or-english/}

\myfootnote{} because the length of numbers in any base grows logarithmically, and logarithms to any two bases are always proportional to each other, so are the growth rates of numbers in any two bases. so in an ideal situation, if every base could have symbols that are proportionally more lightweight or packed, depending on the size of the base, then their relative lengths would be entirely dependent on their {\it radix economy}, which is something we'll also get to.

\myfootnote{} a decent ASCII approximation of this system is \verb#.# for 0, and \verb#|# for 1. for grouping, you can put spaces to delimit groups, or even use \underline{underlining} if you're feeling fancy. handwritten forms, if distinct from the ``printed'' ones at all, could look something like the letters u and \textturnm.

\myfootnote{} for just one implication, the single most cited paper in psychology ({\it The Magical Number Seven, Plus Or Minus Two}) states that the optimal number of items to have in short-term memory falls somewhere around seven. many binary numbers will have more digits than that, but just reimagining them as octal indeed seems to make them a lot easier to keep in memory, even though no actual base conversion is required and the information content is still exactly the same.

\myfootnote{} in general, whenever you're using digits that cannot be instantly broken down and arbitrarily regrouped, this benefit is lost. this also applies to digits whose design is inspired by the bits, or where the bits are stacked vertically, giving a privilege to one specific grouping scheme over all others, like these: \\
\myurl{https://arxiv.org/ftp/arxiv/papers/1707/1707.03751.pdf} \\
\myurl{https://en.wikipedia.org/wiki/Bibi-binary}

\myfootnote{} in base $b^n$, the probability that a $k$-digit number will have \emph{less} than $n \times k$ digits in base $b$ rapidly converges to $1/b$ as $n$ increases.

\myfootnote{} Misali was born in ``1JI'', three niftimal digits; or ``13130'', not six, but \emph{five} seximal digits.

\myfootnote{} the genetic code represents the roughly twenty amino acids using triplets of nitrogen bases, of which there are four kinds. four cubed is 64, the smallest power of four sufficient to represent all amino acids distinctly. the same number of combinations could be represented using sextuplets if there were only two kinds of bases -- but \emph{quintuplets} would already be enough. so barring biological implications such as safety of random mutations, a binary genetic code would be more efficient!

\myfootnote{} Ithkuil uses base one hundred: \\
\myurl{http://ithkuil.net/newithkuil_13_numbers.htm}

\end{document}