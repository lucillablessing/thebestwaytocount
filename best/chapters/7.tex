\documentclass[../best.tex]{subfiles}

\begin{document}

\mychapter[seven]{Mersenne primes, Fermat primes, \\ and the truth about two and six}

this is it. we've hit rock bottom. at every step we've confronted more aspects of the way we count, and it seems there isn't much more to see. everything, from size to arithmetic, from counting to factors, points to the conclusion that being the smallest base is a strength; but to reach that conclusion, we had to untangle assumptions, break conventions and think twice on almost every occasion.

counting in binary requires relearning a lot of what we know about counting, but for a reward that's well worth it. and maybe this is why we see so many proposals of a dozenal or hexadecimal system, but hardly anyone has had the audacity to propose a binary one. and we've barely scratched the surface. we missed binary tally marks,\myfootnote{} balanced ternary,\myfootnote{} two's complement notation,\myfootnote{} a binary monetary system,\myfootnote{} binary timekeeping,\myfootnote{} the dot notation in music,\myfootnote{} and so much more.

this final chapter is not a reprise of all the points we've made so far as much as it is a philosophical exploration of how binary permeates the fabric of existence itself. it's not a case for binary being efficient and useful as much as it is a case for binary being fundamental and natural. if you don't care much for questions like these, then there's nothing more beyond this point, but maybe you can stick around anyway.

\sectionbreak

the Mersenne primes are a famous subset of primes all one less than some power of two. remarkably, the exponent of this power of two must itself be prime; if it was composite, say, 6, we could immediately find a repeating pattern in the binary expansion of $2^6 - 1$ and therefore factorize it. the predecessors of the powers of any other base can never be prime, because they would all repeat a digit greater than 1, and so every such number is divisible at least by that digit.\myfootnote{}

another famous subset are the Fermat primes, all one \emph{more} than a power of two; this time, the exponent must itself be a power of two. otherwise, it would have an ``odd part'', and you could find some alternating sum test which its binary expansion would pass by making the two ones cancel out. this time, other bases can generate Fermat primes -- 37 is a seximal Fermat prime, 101 a decimal Fermat prime -- but no matter the base, the \emph{exponent} must be a power of specifically two.\myfootnote{}

of course, the powers of two occur throughout mathematics with astonishing frequency\myfootnote{} -- but more meaningfully, they do so for the same reason that binary triumphs over all other bases: because 2 is the smallest integer greater than 1.

six, on the other hand, may have lots of spectacular properties: it's an antiprime, a triangular number, a perfect number, and many more all at once -- but most of those are no more than arithmetical curiosities, nuggets of trivia from a book on recreational mathematics; and certainly having no connection to the value of seximal as a base.

so, yes, six may be a perfect number, but \emph{all} the currently known perfect numbers are the product of a power of two and one less than the next power of two; and while no power of two is a perfect number, \emph{all} the powers of two are the \emph{only} currently known \emph{almost}-perfect numbers.

six may be the integer part of the true circle constant.\myfootnote{} but two is the integer part of the exponential constant $e$, and this is more than just a coincidence: the same role that $e^x$ has in standard infinitesimal calculus is played by $2^x$ in ``Newton's calculus'' of discrete steps of change.\myfootnote{}

six may be a convenient number to count up to, but so is \emph{every} power of two, because of how much they reduce counting to just a feeling of rhythm. six may be a better base for orders of magnitude than ten, but two is even better, giving an estimate of size that's often good enough on its own, without a mantissa.

six may be an antiprime, but so is two. it may be a superior highly composite number, but so is two. both two and six are the only antiprimes that are also primorials, but two is the only antiprime that's also... a prime.

with how ubiquitous the powers of two are -- there's even a whole \emph{game} about them\myfootnote{} -- they may very well be the most important integer sequence. even Misali seems to agree, as the powers of two are the \emph{very first} entry on the list of things worth memorizing if switching to seximal.\myfootnote{}

two is so important that many languages have an entire grammatical number for it: the dual.\mytranscript{1:07:15} no, wait, \emph{this} one.\myfootnote{} and remind me what number Toki Pona can count up to? not six, that's for sure.\mytranscript{1:07:19}

perhaps cells divide into two because 2 is the smallest integer greater than 1 -- because duplication is the least effort that produces growth. chromosomes and DNA base strands are paired because the smallest number of spares you can have is one -- ``duplication'' and ``copying'' are basically synonyms.

perhaps electrons and positrons form a pair because 2 is the smallest integer greater than 1 -- if there was only one, there'd be no conservation of charge; any number greater than 1 works, of which 2 is the smallest. duality is the simplest possible system of opposites -- maybe that's why it's so common.

binary counting is fundamental to rhythm,\mytranscript{1:08:00} but that's not its only musical aspect: it turns out \emph{pitch} is binary too. raising a note by an octave doubles its frequency, so octave equivalence, such a big part of our music perception that we give octave-equivalent notes the same names, is just the equivalence of binary representations that differ by at most a bit shift. so you could say our ears have been counting in binary this whole time. perhaps that too is the case because 2 is the smallest integer greater than 1 -- because $2:1$ is the simplest integer ratio after $1:1$.\myfootnote{}

the wonders of binary could go on, but this story about binary ends here. maybe you're convinced that binary is the best way to count. maybe not. maybe you're not sure yet, but you came out knowing more than you did before.

\newpage

numeral systems are important to nerds, but what's more important is coming to your own conclusions instead of accepting one person's opinion as fact. in the end, none of this really matters, because no numeral system is better than any other at depicting numbers as the abstract entities they really are, and choosing one over another is just an artifact of our human desire to compete, to improve, to optimize.

but in a way, that's also what makes this discussion so valuable and beautiful. if we could refer to numbers directly, no numeral system would ever exist, and maybe we'd never discover the deep and wonderful math behind it all. there'd never be an exchange of opinions or a trip down rabbit holes. it's our own limitations that make us strong, and understanding that is what really counts.\myfootnote{} ...in binary, that is.\mytranscript{1:11:48}

\end{document}