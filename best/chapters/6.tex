\documentclass[../best.tex]{subfiles}

\begin{document}

\mychapter[six]{how do I say these numbers with my mouth}

at the end of the day, counting systems don't exist in a vacuum: they exist within languages. this topic is intentionally at the very end: to demonstrate that binary is a big deal even in the absence of a system of number names, rather than introduce one right at the start -- and also because even here, there's a surprising amount of depth in how words for numbers work.

at first, binary seems like a nightmare for pronouncing numbers: surely all this ``one zero one zero'' nonsense would drive you crazy. but most counting systems don't just read off all the digits of a number in sequence. instead, the most common approach seems to be to follow each nonzero digit with a word for the power of the base that it represents -- so that, for instance, ``three two five'' becomes ``three hundred two ten five''.\mytranscript{55:56} even though this seems more costly, it has several advantages, such as letting you know a number's rough size right away, and being shorter for numbers with many zeros.\myfootnote{}

if we were to follow this method, what we'd need is words for each digit in the base, and words for every power of the base. but, again, it still seems that binary is the worst base to design such a counting system for. its powers are the most dense, so you need lots of words to cover the same range where you'd get by with far fewer words in other bases.

but we've overlooked something crucial. take a step back and look at decimal. do we \emph{really} have words for every power of ten? we call $10^2$ ``hundred'' and $10^3$ ``thousand''. but what about $10^4$? $10^5$? the next named power of ten is $10^6$, ``million''. and in general, we only give powers of $10^3$ unique names, and fill in the gaps with a kind of sub-base of three.\myfootnote{}

but why stop there? you could name powers of $10^3$ up to $10^9$, and then start only naming powers of $10^9$. in fact, you can do it smarter: instead of naming \emph{three} powers before skipping, you can do just \emph{two}. in other words, the only powers of ten you really need to name are those where the exponent is a \emph{power of two}. this was the concept behind the Chinese long scale,\myfootnote{} or Donald Knuth's -yllion system.\myfootnote{}

so in binary, we only need to name... the double powers of two. the super-powers of two. the meta-powers of two. whatever you wanna call them. and it turns out that naming only these powers makes it so we actually need \emph{fewer} names in the long run than larger bases: decimal, for instance, would need sixteen number names to exceed a googol, while binary only needs ten.

if we analyze the number of distinct words necessary up to a number $n$, we can see why that's the case. a higher base needs more digit names than a lower base, with a disadvantage equal to their difference. if we name \emph{all} the powers of each base, then the higher base more than makes up for this initial disadvantage. that's because the number of power names required to get to $n$ is proportional to the logarithm of $n$, so the advantage of using a higher base at $n$ is equal to the difference of the logarithms of $n$ in the two bases, and will eventually -- for a large enough value of $n$ -- always outweigh the initial disadvantage. however, if we only name \emph{power-of-two} powers of each base, \emph{now} the number of power names up to $n$ is proportional to the \emph{double} logarithm of $n$. and now the higher base is no longer \emph{increasingly} better for bigger values of $n$; it now only has a \emph{constant} advantage, being the difference of the double logarithms of the bases. but linear functions grow way faster than logarithms, let alone \emph{double} logarithms; so the disadvantage from needing more digit names will always outweigh the advantage of needing one or two fewer power names. surprisingly, efficiency again increases as the base decreases, reaching its maximum for binary.\myfootnote{}

right, but what should the names for the double powers of two be? well, here's the best part. we don't need to invent any -- computer scientists have already done it for us: {\it two}, {\it four}, {\it hex}, {\it byte}, {\it short}, {\it int}, {\it long}, perhaps {\it overlong}, and finally, 2 to the power of byte could be {\it byteplex}.\myfootnote{} just these few names, most of which are one syllable, are enough to reach the number of particles in the visible universe.\myfootnote{}

in binary specifically, we get an additional unexpected advantage from having the exponents themselves follow a binary pattern: orders of magnitude become much easier to use. to represent \emph{any} power of two, just list the double powers of two under its bits and pronounce the ones underneath a 1 bit, in ascending order. in decimal, the exponents aren't themselves decimal, but a kind of base 3; neither are they seximal in seximal, but a kind of base 4. meanwhile, in binary, these two systems are one and the same.

okay, how do we convert to and from this system? to pronounce a number, break it up at its largest double power of two, and represent the left and right sides recursively, with the name of the double power of two between them. parsing a number works the other way: find the largest double power of two in your word string, and your number is equal to the left side times that double power of two, plus the right side. both of those algorithms may seem arcane -- they involve recursion, after all\myfootnote{} -- but they're more for completeness; in reality there's a clear pattern to these number names, one that's very easy to pick up on.\mytranscript{1:00:17}

as it stands, there still is one problem with this system: many names for small numbers are pretty long. but there's an easy way to fix that: we already have names for such numbers, so we can carefully reintroduce them in places. {\it three} is a good choice, as a shortening of the very common combination {\it two one}, essentially making the system quaternary (wait a second. wasn't base 4 supposed to be bad? not in this case; read the footnote if you're curious).\myfootnote{} we can also give some small powers of two their own names even if they're not double powers of two: {\it eight} comes to mind.\myfootnote{} as does eight squared,\myfootnote{} since there already is a word for that very number in common parlance.\mytranscript{1:00:53}

now, sit back and enjoy, and watch some names of numbers scroll by.

\end{document}