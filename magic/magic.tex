\documentclass[a4paper, 12pt]{report}
\usepackage{magic}

\title{the best way to count \\ \Large bonus: magic sequences}
\author{Lucilla}
\date{}

\begin{document}

\vfill \maketitle \vfill \newpage

\begin{definition}[floor, fract]
	the \emph{floor function} $x \mapsto \floor{x}$ is defined as the largest integer less than or equal to $x$. \\ it satisfies $\floor{x} = x$ for integer $x$ and $x - 1 < \floor{x} < x$ otherwise. \\ this function is idempotent: $\floor{\floor{x}} = \floor{x}.$ \bigskip

	the \emph{fractional part function} $x \mapsto \fract{x}$ is defined as $x - \floor{x}$. \\ it satisfies $\fract{x} = 0$ for integer $x$ and $0 < \fract{x} < 1$ otherwise. \\ this function is idempotent: $\fract{\fract{x}} = \fract{x}$. \\ it is also periodic with period $1$; if $n$ is an integer, then $\fract{x + n} = \fract{x}$.
\end{definition} \bigskip

\begin{definition}[div, mod]
	the \emph{Euclidean division functions} \tdiv{} and \tmod{} are defined as follows:

	\begin{itemize}
		\item $a \fdiv b = \floor{a / b}$;
		\item $a \fmod b = b \cdot \fract{a / b}$.
	\end{itemize}

	\tdiv{} and \tmod{} satisfy the \emph{Euclidean division theorem}: \\ for any integer $a$ and natural number $b$ with $b \neq 0$, there exist integers $q$ and $r$ such that $0 \leq r < b$ and $a = q \cdot b + r$, namely $q = a \fdiv b$ and $r = a \fmod b$. \medskip

	the simultaneous application of \tdiv{} and \tmod{} is denoted \tdivmod; the statement $a \fdivmod b = (q, r)$ is to be read as $a \fdiv b = q \ \wedge \ a \fmod b = r$.
\end{definition} \bigskip

\begin{definition}[magic sequence]
	let $b$ be a natural number (the \emph{base}); $b > 1$. \\
	let $n$ be a natural number, $n > 1$. \\
	let $s$ be a natural number, $1 \leq s < n$. \medskip

	the \emph{magic sequence} for $n$ in \emph{base} $b$ with \emph{offset} $s$ (usually $s = 1$) is an infinite sequence of integers $\m{mag}_i$ ($i \geq 0$) defined recursively as:

	\begin{itemize}
		\item $\m{mag}_0 = s$;
		\item $\m{mag}_{i + 1} = (b \cdot \m{mag}_i) \fmod n$.
	\end{itemize}

	the corresponding \emph{quotient sequence} is the sequence $\m{div}_i$ ($i \geq 1$) defined as \\ $\m{div}_{i + 1} = (b \cdot \m{mag}_i) \fdiv n$.
\end{definition} \bigskip

example of a magic sequence with $b = 6, n = 11, s = 1$:

\begin{itemize}
	\item $\m{mag}_0 = 1$;
	\item $(1 \cdot 6) \fdivmod 11 = (0, 6) \ \Rightarrow \ \m{div}_1 = 0, \ \m{mag}_1 = 6;$
	\item $(6 \cdot 6) \fdivmod 11 = (3, 3) \ \Rightarrow \ \m{div}_2 = 3, \ \m{mag}_2 = 3;$
	\item $(3 \cdot 6) \fdivmod 11 = (1, 7) \ \Rightarrow \ \m{div}_3 = 1, \ \m{mag}_3 = 7;$
	\item $(7 \cdot 6) \fdivmod 11 = (3, 9) \ \Rightarrow \ \m{div}_4 = 3, \ \m{mag}_4 = 9;$
	\item \ldots
\end{itemize}

\newpage

\section*{magic sequence divisibility test}

\begin{lemma}
	let $u, v$ be integers, and let $n$ be a natural number, $n > 1$. then:
	\begin{itemize}
		\item $(u + v) \fmod n = ((u \fmod n) + (v \fmod n)) \fmod n$;
		\item $(u \cdot v) \fmod n = ((u \fmod n) \cdot (v \fmod n)) \fmod n$.
	\end{itemize}
\end{lemma}

\begin{proof}
	the first statement is equivalent to
	\begin{equation*}
		\fract{x + y} = \fract{\fract{x} + \fract{y}}
	\end{equation*}
	by setting $x := \frac{u}{n}, \ y := \frac{v}{n}$; observe that
	\begin{equation*}
		  \fract{\fract{x} + \fract{y}}
		= \fract{x - \floor{x} + y - \floor{y}}
		= \fract{x + y - (\floor{x} + \floor{y})};
	\end{equation*}
	noting that $(\floor{x} + \floor{y})$ is an integer completes the equality. \bigskip

	the second statement is equivalent to
	\begin{equation*}
		  \fract{\frac{u}{n} \cdot \frac{v}{n} \cdot n}
		= \fract{\fract{\frac{u}{n}} \cdot \fract{\frac{v}{n}} \cdot n}
	\end{equation*}
	where the right-hand side can be expanded as
	\begin{align*}
		    \fract{\fract{\frac{u}{n}} \cdot \fract{\frac{v}{n}} \cdot n}
		& = \fract{
			\left(\frac{u}{n} - \floor{\frac{u}{n}}\right) \cdot
			\left(\frac{v}{n} - \floor{\frac{v}{n}}\right) \cdot n
		} \\
		& = \fract{
			\left(
			         \frac{u}{n}  \cdot        \frac{v}{n}
			-        \frac{u}{n}  \cdot \floor{\frac{v}{n}}
			- \floor{\frac{u}{n}} \cdot        \frac{v}{n}
			+ \floor{\frac{u}{n}} \cdot \floor{\frac{v}{n}}
			\right) \cdot n
		} \\
		& = \fract{
			\frac{u}{n} \cdot \frac{v}{n} \cdot n +
			\left(
			      \floor{\frac{u}{n}} \cdot \floor{\frac{v}{n}} \cdot n
			    - u \cdot \floor{\frac{v}{n}}
			    - v \cdot \floor{\frac{u}{n}}
			\right)
		}
	\end{align*}
	and again noting that
	\begin{equation*}
		\left(
			\floor{\frac{u}{n}} \cdot \floor{\frac{v}{n}} \cdot n
			- u \cdot \floor{\frac{v}{n}}
			- v \cdot \floor{\frac{u}{n}}
		\right)
	\end{equation*}
	is an integer.
\end{proof}

\begin{corollary}
	variants where only one variable has had \tmod{} applied to it, namely
	\begin{equation*}
		\begin{array}{c}
			(u + v) \fmod n = (u + (v \fmod n)) \fmod n \\
			(u \cdot v) \fmod n = (u \cdot (v \fmod n)) \fmod n
		\end{array}
	\end{equation*}
	follow with a similar proof. alternatively, they can be derived by applying the normal variant twice, then using the idempotence of \tmod{} (which follows from the idempotence of the fractional part function) on one of the variables, then applying the normal variant backwards.
\end{corollary}

\newpage

\begin{lemma}
	let $\m{mag}_i$ be the magic sequence of $n$ in base $b$ with offset $s$. then
	\begin{equation*}
		(s \cdot b^i) \fmod n = \m{mag}_i
	\end{equation*}
	for all $i \geq 0$.
\end{lemma}

\begin{proof}
	induction for $i$.

	let $i = 0$, then we have $s \fmod n = s$, which is true because $1 \leq s < n$. \smallskip

	suppose the statement is true for some $i$, then
	\begin{align*}
		    \m{mag}_{i + 1}
		& = (b \cdot \m{mag}_i) \fmod n \\
		& = (b \cdot (s \cdot b^i) \fmod n) \fmod n \\
		& = (b \cdot s \cdot b^i) \fmod n \\
		& = (s \cdot b^{i + 1}) \fmod n,
	\end{align*}
	thus the statement is true for $i + 1$, and hence for all $i \geq 0$.
\end{proof}

\begin{definition}[base-$b$ representation]
	let $b$ be a natural number (the \emph{base}); $b > 1$. \\
	let $\m{num}$ be a natural number, $\m{num} \geq 0$. \medskip

	the \emph{base-$b$ representation} of $\m{num}$ is a sequence $d_i$ ($i \geq 0$) which satisfies $0 \leq d_i < b$ for all $i$, only finitely many $d_i$ are nonzero, and
	\begin{equation*}
		\sum_i b^i \cdot d_i = \m{num}.
	\end{equation*}

	the sequence $d_i$ exists and is unique. $d_i$ are called the \emph{digits} of $\m{num}$ in base $b$.
\end{definition}

\begin{theorem}[magic sequence divisibility test]
	let $\m{mag}_i$ be the magic sequence of $n$ in base $b$ with offset $s = 1$. \\
	let $\m{num}$ be a natural number, and let $d_i$ be its digits in base $b$. \\
	then
	\begin{equation*}
		\m{num} \fmod n = \Big( \sum_i \m{mag}_i \cdot d_i \Big) \fmod n.
	\end{equation*}
\end{theorem}

\begin{proof}
	\begin{align*}
		    \m{num} \fmod n
		  = \Big( \sum_i b^i \cdot d_i \Big) \fmod n
		& = \sum_i \big( (b^i \cdot d_i) \fmod n \big) \fmod n \\
		& = \sum_i \big( ((b^i \fmod n) \cdot d_i) \fmod n \big) \fmod n \\
		& = \sum_i \big( (\m{mag}_i \cdot d_i) \fmod n \big) \fmod n \\
		& = \Big( \sum_i \m{mag}_i \cdot d_i \Big) \fmod n.
	\end{align*}
\end{proof}

\section*{magic sequence fractions}

\begin{lemma}
	let $\m{mag}_i$ be the magic sequence of $n$ in base $b$ with offset $s$, \\
	and let $\m{div}_i$ be the corresponding quotient sequence. then
	\begin{equation*}
		\frac{s \cdot b^i}{n}
		- b^{i - 1} \cdot \m{div}_1
		- b^{i - 2} \cdot \m{div}_2
		- \ldots
		- \m{div}_i
		= \frac{\m{mag}_i}{n}
	\end{equation*}
	for all $i \geq 0$.
\end{lemma}

\begin{proof}
	induction for $i$.

	let $i = 0$, then we have
	\begin{equation*}
		\frac{s \cdot b^0}{n} = \frac{s}{n} = \frac{\m{mag}_0}{n}.
	\end{equation*}
	\smallskip

	suppose the statement is true for some $i$, then \bigskip
	\begin{align*}
		&   \hphantom{=\ } \frac{s \cdot b^{i + 1}}{n}
		  - b^{i}     \cdot \m{div}_1
		  - b^{i - 1} \cdot \m{div}_2
		  - \ldots
		  - b         \cdot \m{div}_i
		  - \m{div}_{i + 1} \\
		& = b \cdot \left(
		    \frac{s \cdot b^i}{n}
		  - b^{i - 1} \cdot \m{div}_1
		  - b^{i - 2} \cdot \m{div}_2
		  - \ldots
		  - \m{div}_i
		  \right)
		  - \m{div}_{i + 1} \\
		& = \frac{b \cdot \m{mag}_i}{n} - \m{div}_{i + 1} \\
		& = \frac{b \cdot \m{mag}_i}{n} - (b \cdot \m{mag}_i) \fdiv n \\
		& = \frac{b \cdot \m{mag}_i}{n} - \floor{\frac{b \cdot \m{mag}_i}{n}}
		  = \fract{\frac{b \cdot \m{mag}_i}{n}} \\
		& = \frac{(b \cdot \m{mag}_i) \fmod n}{n}
		  = \frac{\m{mag}_{i + 1}}{n},
		\end{align*}
	thus the statement is true for $i + 1$ and hence for all $i \geq 0$.
\end{proof}

\begin{definition}[base-$b$ fractional expansion]
	let $b$ be a natural number (the \emph{base}); $b > 1$. \\
	let $\m{ratio}$ be a real number, $0 \leq \m{ratio} < 1$. \medskip

	the \emph{base-$b$ fractional expansion} of $\m{ratio}$ is a sequence $r_i$ ($i \geq 1$) which satisfies \\ $0 \leq r_i < b$ for all $i$ and
	\begin{equation*}
		\m{ratio} - \frac{1}{b^k} < \sum_{i \leq k} \frac{r_i}{b^i} \leq \m{ratio}
	\end{equation*}
	for all $k \geq 1$. \medskip

	the sequence $r_i$ exists and is unique. $r_i$ are called the \emph{digits} of $\m{ratio}$ in base $b$, and it holds that
	\begin{equation*}
		\sum_i^{\infty} \frac{r_i}{b^i} = \m{ratio}.
	\end{equation*}
	(the second condition is necessary for uniqueness. without it, some real numbers have two different fractional expansions, e.g. $\frac{1}{5} = 0.1999\ldots = 0.2000\ldots$ in base ten. with this definition, only $0.2000\ldots$ is a valid fractional expansion.)
\end{definition}

\begin{theorem}[magic sequence fractions]
	let $\m{mag}_i$ be the magic sequence of $n$ in base $b$ with offset $s$, \\
	and let $\m{div}_i$ be the corresponding quotient sequence. \\
	let $r_i$ be the digits of $s/n$ in base $b$. \\
	then $\m{div}_i = r_i$ for all $i \geq 1$.
\end{theorem}

\begin{proof}
	induction for $i$.

	first let $i = 1$. suppose $\frac{s}{n} = \frac{r_1}{b}$ exactly.
	then $r_1 = \frac{s \cdot b}{n}$. but $r_1$ must be an integer,
	so take
	\begin{equation*}
		  \floor{\frac{s \cdot b}{n}}
		= (s \cdot b) \fdiv n
		= (b \cdot \m{mag}_0) \fdiv n
		= \m{div}_1
	\end{equation*}
	instead.
	the inequality $x - 1 < \floor{x} \leq x$ for the floor function then implies the inequality
	\begin{equation*}
		\frac{s}{n} - \frac{1}{b} < \frac{r_1}{b} \leq \frac{s}{n}.
	\end{equation*} \bigskip

	now suppose $\m{div}_j = r_j$ for all $j \leq i$. suppose
	\begin{equation*}
		  \frac{s}{n}
		= \frac{r_1}{b}
		+ \frac{r_2}{b^2}
		+ \ldots
		+ \frac{r_i}{b^i}
		+ \frac{r_{i + 1}}{b^{i + 1}}
	\end{equation*}
	exactly; this implies
	\begin{equation*}
		  \frac{s}{n}
		- \frac{\m{div}_1}{b}
		- \frac{\m{div}_2}{b^2}
		- \ldots
		- \frac{\m{div}_i}{b^i}
		= \frac{r_{i + 1}}{b^{i + 1}}.
	\end{equation*}
	in that case
	\begin{equation*}
		  r_{i + 1}
		= b \cdot \left(
		  \frac{s \cdot b^i}{n}
		- b^{i - 1} \cdot \m{div}_1
		- b^{i - 2} \cdot \m{div}_2
		- \ldots
		- \m{div}_i
		\right),
	\end{equation*}
	which by the lemma above is equal to $b \cdot \frac{\m{mag}_{i + 1}}{n}$.
	but $r_{i + 1}$ must be an integer,
	so take
	\begin{equation*}
		  \floor{\frac{b \cdot \m{mag}_{i + 1}}{n}}
		= (b \cdot \m{mag}_{i + 1}) \fdiv n
		= \m{div}_{i + 1}
	\end{equation*}
	instead; and again the floor function inequality implies
	\begin{equation*}
		\frac{s}{n} - \frac{1}{b^{i + 1}} < \sum_{j \leq i + 1} \frac{r_j}{b^j} \leq \frac{s}{n}.
	\end{equation*}
\end{proof}

\end{document}
